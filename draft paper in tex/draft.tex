\documentclass[conference]{IEEEtran}
\IEEEoverridecommandlockouts
\usepackage{cite}
\usepackage{amsmath,amssymb,amsfonts}
\usepackage{algorithm}
\usepackage{algorithmic}
\usepackage{graphicx}
\usepackage{textcomp}
\usepackage{xcolor}
\usepackage{booktabs}
\usepackage{multirow}
\usepackage{tikz}
\usetikzlibrary{shapes.geometric, arrows, positioning}
\usepackage{pgfplots}
\pgfplotsset{compat=1.18}
\def\BibTeX{{\rm B\kern-.05em{\sc i\kern-.025em b}\kern-.08em
    T\kern-.1667em\lower.7ex\hbox{E}\kern-.125emX}}
\usepackage{url}
\begin{document}

\title{Optimizing Classroom Space Allocation Using MILP and Constraint Programming: A Comparative Study of Open-Source Solvers for University Timetabling}

\author{\IEEEauthorblockN{Mazo, Kenji U.}
\IEEEauthorblockA{\textit{College of Computing and Multimedia Studies} \\
\textit{Camarines Norte State College}\\
Talisay, Philippines \\
kenjimazo@gmail.com}
\and
\IEEEauthorblockN{Quierra, Janero Marco M.}
\IEEEauthorblockA{\textit{College of Computing and Multimedia Studies} \\
\textit{Camarines Norte State College}\\
Talisay, Philippines \\
email@example.com}
}

\maketitle

\begin{abstract}
This study presents an optimization model for university classroom allocation, comparing free and open-source solvers as alternatives to commercial software. Addressing the challenges of the College of Computing and Multimedia Studies (CCMS) at Camarines Norte State College, the research formulates the timetabling problem using a session-based scheduling system with strict room category constraints. The system implements a full \textbf{2 strategies $\times$ 3 solvers} matrix: Program-by-Program Sequential and Global strategies combined with CBC (MILP), HiGHS (MILP), and OR-Tools CP-SAT (Constraint Programming). This design allows direct comparison between MILP and CP approaches. The Program-by-Program strategy achieves optimal solutions in 35--68 seconds across all solvers, while the Global strategy with OR-Tools CP-SAT solves the full problem in 37 seconds. The model handles approximately 300,000 binary variables for a dataset of 18 student blocks, 50+ courses, and 17 rooms. A GUI-based tool allows administrators to configure parameters for capacity planning. Feasibility analysis confirms sufficient resources with 64\% lecture and 52\% lab utilization. Results demonstrate that open-source solvers can effectively handle real-world timetabling in resource-constrained institutions.
\end{abstract}

\begin{IEEEkeywords}
classroom scheduling, binary integer programming, constraint programming, open-source solvers, optimization, resource allocation, higher education, Philippine universities
\end{IEEEkeywords}

\section{Introduction}

Higher Education Institutions (HEIs) in developing countries have one of the biggest challenges when it comes to the efficient use of classroom resources as a result of rapid growth in enrollment relative to the development of their infrastructure. Camarines Norte State College's College of Computing and Multimedia Studies (CCMS) is no exception. As a result of this challenge, CCMS is growing but continues to be limited by its current lab facilities; specifically, CCMS currently owns only nine rooms in the new three-story building, as well as three labs in the old building and shares the other rooms with other programs.

In addition to its limited lab facilities, CCMS is also currently serving approximately 850 students in both IT and IS programs. Due to these limited facilities, CCMS still utilizes a manual process to schedule classes whereby the assignment of rooms and times are made by hand. As a result of the limitations of this manual process, students are forced to endure lengthy wait times for assignments, inefficient use of classrooms, and the occurrence of scheduling conflicts. Thus, an efficient and optimized scheduling system is needed to meet the projected growth in enrollment which is expected to reach 1230 students by 2030. This aligns with broader efforts to digitize management systems in Philippine HEIs to enhance performance and efficiency \cite{spms2025}.

As documented through research, manual processes for scheduling classes in universities result in inefficient use of available classroom space. Specifically, studies conducted by researchers such as Austero et al., \cite{austero2022optimizing} demonstrated that optimization utilizing a Heuristically Enhanced Whale Optimization Algorithm (HEWOA) was able to reduce the number of rooms required for 1700 classes from more than 120 to 91 (p. 560). Furthermore, Schininà \cite{schinina2024timetabling} used integer linear programming (ILP) to create a curriculum based timetabling solution for the University of Groningen which resulted in a reduction of classroom usage from 26 to 18-22 for different academic blocks (pp. 43-44) resulting in zero conflicts and minimized waste of unused space. These inefficiencies can further exacerbate existing problems within the Philippine education system, including gaps in infrastructure which limit the ability of STEM programs to grow and ultimately limit the growth of the country's ICT sector \cite{navarro2022school,bayudan2024expansions}.

This paper presents a binary integer linear programming model. The model adapts optimization methods that universities with limited resources already use. It fits the exact rules that Philippine higher education institutions must follow. Its objective function maximizes the count of course components that receive a conflict free timetable.

The study tests three free, open source solvers - CBC, HiGHS besides Google OR-Tools CP-SAT - as replacements for costly commercial solvers like Gurobi and CPLEX. Academic licenses for those commercial products cost an institution between ten thousand and fifty thousand dollars each year. Commercial scheduling packages also charge eight thousand five hundred to twelve thousand dollars annually \cite{ofcourse2023}. The open source tools cost nothing and allow local modification - they suit institutions in developing countries that operate under tight budgets.

The work answers the problem noted by Aygül et al. \cite{aygul2025}: mathematically sound models or commercial packages often stay unused because universities lack technical staff, money or flexible systems. Literature on educational facility planning states that room utilization should reach forty three to fifty nine percent depending on room size - some universities aim for seventy five to eighty five percent when possible \cite{cornell2022classroom}. This study adopts the seventy five to eighty five percent range as its benchmark \cite{lindahl2018strategic,oude2019practices}, because settings with scarce resources must use every room efficiently.

This study contributes to the literature in four areas: (1) it establishes a mathematical model that meets the needs of CCMS and the constraints of typical Philippine Higher Education Institutions (HEIs), specifically by maintaining strict room category separation while still allowing for session-based scheduling; (2) it develops an exhaustive evaluation of three Free and Open-Source solvers (CBC, HiGHS, and OR-Tools CP-SAT) applied to a real-world timetabling problem at a university; (3) it designs a dynamic model that can be used to update parameters (such as student enrollment, courses offered, etc.) that are defined via CSV files and do not require updating of code; and (4) it validates its model with actual curriculum data from CCMS for IT and IS programs over all four years of study, providing evidence that the model is scalable and applicable to universities in the Philippines.

\section{Literature Review}

Recent studies about this topc focuses on mutli-obkective optimations and constraints specific to a certain context. The operations research sector has studiedof the classroom scheduling problem. This paper covers studies in three sections: space utlization, timetabling methods, and developing country applications.

\textbf{Timetabling Optimization Methods}

To support the feasability of our approach which involoves around 300,000 binary variables for the full dataset, Tan et al. \cite{tan2021} reviewed 47 studies about timetabling, and learned that binary intger programming (BIP) effective for problems with decision variables of moderate amount.

Following the paper of Ceschia et al. \cite{ceschia2023} which provided validation methods for education timetabling, which includes resouce availability conditions, benchmark problem formulations, robustness testing under varying enrollment, and state of the art solution quality assemets. We tested our model with the same constraints to ensure it meet the context and other scenarios as well.

One study used ILP to solve a timtabling by doing it per curriculum at the University of Groningen, this study reduced the rooms from 26 to 18-22 across 350 blocks in their university Schininà \cite{schinina2024timetabling}. This is very relevant to this study as it demonstrate the real world impact of using ILP, it shows that optimization can reduce the requirements for resource while making it feasible.

Davison et al. \cite{davison2024hybrid} set out a multi objective binary programme that builds the university timetable while it accounts for both in person and online teaching. The model balances the goals of multiple groups plus multiple kinds of rooms and equipment. It shows that a binary programme can still deliver a timetable when the aims conflict - grant as many module requests as possible, keep clashes but also gaps low and respect the staff preference for either face-to-face or online delivery.

Alnaji et al. \cite{alnaji2024optimizing} give a numerical model that decides how many lecturers Hafr Al Batin University needs. The model tracks how student numbers change, what quality of teaching the college expects as well as which degrees the institution offers. It fixes the following staff-to-student ratios - one lecturer for every twenty students in health and engineering, one for every thirty in the sciences or one for every forty in remaining fields (pp. 91 - 92). The study shows that an optimisation formula can turn limited staff into a plan that rests on clear data.

\textbf{Constraint Programming and OR-Tools}

Although Integer Linear Programming is still the main method used in the literature, more researchers now turn to Constraint Programming solvers like Google OR-Tools CP-SAT. Those solvers handle non linear constraints well and search through large spaces efficiently. In the detailed technical review by Gu et al. \cite{gu2025}, the authors observe that modern CP solvers that use Lazy Clause Generation - CP-SAT is one example - often beat traditional MIP solvers when the goal is to locate a feasible schedule in heavily constrained university settings.

Practical use of those tools has already been shown. Rabadia et al. \cite{rabadia2025} enlarged Naderi's 2016 framework - applying both CP-SAT and SCIP from Google OR-Tools to the University Course Timetabling Problem. Their tests show that the CP-SAT solver reached proven optimal answers in less than nine seconds for the basic model. The extended SCIP model, which adds real world rules like session patterns building transfers and workload caps, reached proven optimality in about two minutes. This level of speed matters for our work, because it confirms that OR-Tools offers a computationally cheap option for institutions that possess only modest hardware.

\textbf{Space Utilization in Educational Settings}

A matrix from the study of Vermuyten et al. \cite{vermuyten2018integrated} which developed an integrated staff approach for scheduling using compatibility matrices for medical emergency services, and to match personnel skills with shift requirements. This study adapted this approach and applied the compatibility matrix concept in our $R_{jc}$ matrix, matching specialized laboratories to courses that matches it.

By applying a Heuristically Enhanced Whale Optimization Algorithm (HEWOA), Austero et al. \cite{austero2022optimizing} was able to reduce rooms from more than 120 to 91 for 1700 classess, optimizing rooms and faculty at Bicol University. Proving that optimization outperforms manual methods in the context of Philippine HEI.

A study of Lindahl et al. \cite{lindahl2018strategic} which investigated room planning and teaching period allocation along side strategic timetabling decisions, and analyzing how available resources affect the quality of timetables at universities, demonstrates the importance of considering the allocation of resource that impacts educational quality as strategic decision.

To identify the gaps between research and practice, Oude Vrielink et al. \cite{oude2019practices} conducted a systematic review of timetabling practices in HEIs, it also discussed the use of benchmark used across different universities. This work showed the improtance of practical implementaion frameworks that is needed in solving the gap of theoretical optimization and real world constraints.

\textbf{Developing Country Context}

Mtonga et al., \cite{mtonga2021}, built a simple plan that assigns classrooms at the University of Rwanda. They used linear programming to solve the same problems Philippine colleges face - few rooms and more students each year. Their step-by-step method gives the best fit when resources are tight plus other schools with the same limits may copy it.

Navarro,\cite{navarro2022school}, describes the national level resource deficits in Philippine education. It was documented that rural school settings are significantly limited by resource availability and that enrollment growth rates exceed the rate of construction of new educational facilities. This environment presents the necessity of optimization to meet national goals of STEM and ICT development.

Bayudan-Dacuycuy,\cite{bayudan2024expansions}, examined how public universities in the Philippines grew. He showed that when campuses expand without new buildings, laboratories or staff, the quality of instruction drops. The same study notes that if growth proceeds quickly and without coordination, the institution wastes the assets it already owns and the standard of teaching falls.

Rules for planning educational space differ from one school to another. Cornell University's classroom guide \cite{cornell2022classroom} states that large lecture halls of 151 seats or more should sit empty 57 percent of the time, while rooms with 1 - 100 seats should sit empty 41 percent of the time - that timetable clashes can be absorbed (p. 5). Work on strategic timetabling \cite{lindahl2018strategic} plus a systematic review \cite{oude2019practices} report that when flexibility is weighed against limited space, universities can push actual use to 75 - 85 percent. Our model adopts this 75 - 85 percent band and reads requirements from a CSV file - that schools in the Philippines with few resources can still apply the tool.

\section{Materials and Methods}

\subsection{Research Design}

This study employs a mixed-methods approach combining quantitative optimization modeling with qualitative stakeholder validation. The research was conducted in three phases: (1) data collection and analysis, (2) model development and implementation, and (3) validation and sensitivity analysis.

\subsection{Data Collection}

Data was collected from multiple sources at CCMS:

\textbf{Enrollment Data:} Official registrar records provided enrollment figures for 850 students distributed across IT (450) and IS (400) programs, organized into 23 blocks with 40 students per block average.

\textbf{Curriculum Data:} The 2022 Revised Bachelor of Science in Information Technology curriculum specified 27 core courses with laboratory requirements. Each course requires 2 lecture hours plus 1 laboratory hour weekly, totaling 3-5 contact hours.

\subsection{Model Formulation}

The classroom allocation problem is set up as a binary integer linear program. 
It contains two kinds of binary decision variables. 

\textbf{Decision Variables:}
The model uses two types of binary decision variables:

\textbf{Pattern Selection Variable:}
\begin{equation}
Y_p = \begin{cases} 
1 & \text{if pattern } p \text{ is chosen for its component} \\
0 & \text{otherwise}
\end{cases}
\end{equation}

\textbf{Session Assignment Variable:}
\begin{equation}
X_{ijk} = \begin{cases} 
1 & \text{if session } i \text{ assigned to room } j \text{ at slot } k \\
0 & \text{otherwise}
\end{cases}
\end{equation}

$i$ stands for a single session or meeting, $j \in J$ represents -- j belongs to the set J of rooms and $k \in K$ -- k belongs to the set K of 30-minute time slots that cover the week.

\textbf{Objective Function:}
Scheduling every course components is the primary obejective. This can be done by maximizing the number of selected patterns:
\begin{equation}
\text{Maximize } Z = \sum_{p \in P} Y_p
\end{equation}

This simple goal is enough because the limits already forbid clashes - fixing room and student-group overlaps and the more patterns we place, the better we use the space. The model puts first a plan that works plus that has no clashes - it treats the cutting of idle time as a lesser aim.

\subsection{Implementation}

The model was implemented using Python 3.13 together with the following libraries:
\begin{itemize}
    \item PuLP 2.7.0 for MIP model formulation
    \item Google OR-Tools 9.8 with CP-SAT solver (primary solver)
    \item HiGHS 1.12.0 as alternative MIP solver
    \item Pandas 2.0 for data processing
    \item Tkinter for GUI development
    \item CSV-based input/output system for data interchange
\end{itemize}

The system implements a full \textbf{2 strategies $\times$ 3 solvers} matrix, providing six distinct solver-strategy combinations:

\begin{itemize}
    \item \textbf{Solvers:} (1) CBC via PuLP (MILP), (2) HiGHS via PuLP (MILP), and (3) Google OR-Tools CP-SAT (Constraint Programming)
    \item \textbf{Strategies:} (1) Program-by-Program Sequential, (2) Global (All-at-Once)
\end{itemize}

Both MILP besides CP-SAT implementations rely on identical \textit{same mathematical constraints} - the choice of patterns, the assignment of sessions, the avoidance of room clashes and the avoidance of clashes among student groups. Each system states those constraints in its own style. MILP states them as linear inequalities that a branch-and-bound routine solves. CP-SAT states them through constraint propagation plus lazy clause generation. Because the two systems share one set of constraints, every pairing among the six test cases tackles the same optimization task - the run time figures are directly comparable.

Tests showed that OR-Tools CP-SAT solved the problems fastest - we set it as the default. In the interface you can change the maximum run time, which starts at 600 seconds and the gap from the best possible answer that you will accept, which starts at five percent.

\subsection{Solution Pipeline}

The overall methodology follows a structured pipeline from raw input data to visualized output. Algorithm~\ref{alg:pipeline} presents the pseudocode for this process.

\begin{algorithm}[htbp]
\caption{University Timetabling Solution Pipeline}
\label{alg:pipeline}
\begin{algorithmic}[1]
\STATE \textbf{Input:} courses.csv, enrollment.csv, rooms.csv
\STATE \textbf{Output:} Conflict-free schedule (CSV + visualization)
\STATE
\STATE \textit{// Phase 1: Data Loading \& Formatting}
\STATE Load and parse CSV files into DataFrames
\STATE Generate course components (course $\times$ type $\times$ group)
\STATE Generate time slots (M-F, 8AM-5PM, 30-min intervals)
\STATE
\STATE \textit{// Phase 2: Constraint Definition}
\STATE Define hard constraints:
\STATE \hspace{1em} - Room conflict: $\leq 1$ session per room per slot
\STATE \hspace{1em} - Group conflict: $\leq 1$ session per group per slot
\STATE \hspace{1em} - Room category: labs $\rightarrow$ lab rooms only
\STATE \hspace{1em} - Pattern selection: exactly 1 pattern per component
\STATE Define soft constraints:
\STATE \hspace{1em} - Same room for multi-meeting sessions
\STATE \hspace{1em} - Same time-of-day preference
\STATE
\STATE \textit{// Phase 3: Pre-solve Feasibility Check}
\STATE Calculate demand (required room-slots)
\STATE Calculate supply (available room-slots)
\IF{demand $>$ supply}
    \STATE \textbf{return} ``Infeasible: insufficient resources''
\ENDIF
\STATE
\STATE \textit{// Phase 4: Model Building \& Solving}
\STATE Build optimization model (BIP or CP-SAT)
\STATE Set objective: Maximize $\sum Y_p$ (scheduled patterns)
\STATE Invoke solver (CBC / HiGHS / OR-Tools)
\STATE
\STATE \textit{// Phase 5: Output Generation}
\IF{solution found}
    \STATE Extract assignments from decision variables
    \STATE Format schedule DataFrame
    \STATE Export to CSV (per-group + combined)
    \STATE Generate visualization (optional GUI)
\ELSE
    \STATE \textbf{return} ``No feasible solution within time limit''
\ENDIF
\end{algorithmic}
\end{algorithm}

\section{Mathematical Model}

Our model is formulated as a binary integer linear program. It is designed to first select an optimal weekly scheduling pattern for each course component (e.g., a lecture or a lab section for a specific student group) and then assign the individual meetings of that pattern to specific rooms and time slots.

\subsection{Sets and Indices}
\begin{itemize}
    \item $C$: Set of course components. A component is a unique combination of a course, a session type (lecture/lab), and a student group.
    \item $P_c$: Set of all valid scheduling patterns for a component $c \in C$.
    \item $S_p$: Set of all individual sessions (meetings) that constitute a pattern $p \in P_c$.
    \item $J$: Set of available rooms.
    \item $K$: Set of discrete 30-minute time slots.
    \item $G$: Set of student groups (e.g., IT-3A).
\end{itemize}

\subsection{Decision Variables}
The model uses two types of binary decision variables:

\textbf{Pattern Selection Variable:}
\begin{equation}
Y_p = \begin{cases} 
1 & \text{if pattern } p \text{ is chosen for its component} \\
0 & \text{otherwise}
\end{cases}
\end{equation}

\textbf{Session Assignment Variable:}
\begin{equation}
X_{ijk} = \begin{cases} 
1 & \text{if session } i \in S_p \text{ is assigned to room } j \\
  & \text{at start time } k \\
0 & \text{otherwise}
\end{cases}
\end{equation}

\subsection{Objective Function}
The primary objective is to ensure that every course component is successfully scheduled. This is achieved by maximizing the number of selected patterns:
\begin{equation}
\text{Maximize } Z = \sum_{c \in C} \sum_{p \in P_c} Y_p
\end{equation}

Since the pattern selection constraint (Equation 6) requires exactly one pattern per component, this objective effectively maximizes the number of successfully scheduled components. The constraints handle conflict avoidance, capacity limits, and compatibility requirements, making additional objective terms unnecessary.

\subsection{Constraints}
The model is subject to the following constraints:

\textbf{Pattern Selection Constraint:} Exactly one pattern must be chosen for each course component.
\begin{equation}
\sum_{p \in P_c} Y_p = 1 \quad \forall c \in C
\end{equation}

\textbf{Session Assignment Constraint:} Each session belonging to a chosen pattern must be scheduled exactly once.
\begin{equation}
\sum_{j \in J} \sum_{k \in K} X_{ijk} = Y_p \quad \forall p \in P_c, \forall i \in S_p, \forall c \in C
\end{equation}

\textbf{Room Conflict Constraint:} At most one session can occupy a given room at any single point in time.
\begin{equation}
\sum_{i \in S} \sum_{k' \in K : k \in [k', k'+d_i-1]} X_{ijk'} \leq 1 \quad \forall j \in J, k \in K
\end{equation}
where $d_i$ is the duration of session $i$ in time slots.

\textbf{Student Group Conflict Constraint:} A student group cannot be scheduled for more than one session at the same time.
\begin{equation}
\sum_{i \in S_g} \sum_{j \in J} \sum_{k' \in K : k \in [k', k'+d_i-1]} X_{ijk'} \leq 1 \quad \forall g \in G, k \in K
\end{equation}
where $S_g$ is the set of all sessions for student group $g$.

\textbf{Different-Day Constraint:} All sessions within a multi-meeting pattern must be scheduled on different days of the week.

\textbf{Capacity and Compatibility Constraints:} A session can only be assigned to a room that has sufficient capacity and is compatible with the course's requirements (e.g., a lab course must be in a lab room).

\subsection{Constraint and Parameter Updates}

Recent refinements to the model introduced stricter constraints and dynamic parameters to better reflect institutional policies:
\begin{itemize}
    \item \textbf{Lunch Break Enforcement:} A hard constraint now explicitly blocks all scheduling between 12:00 PM and 1:00 PM, ensuring a common break period for all students and faculty.
    \item \textbf{PathFit Scheduling:} PathFit courses are now restricted to a single 2-hour block once per week, rather than being split across multiple days.
    \item \textbf{Practicum Handling:} Practicum courses (IT 128, IS 404) with excessive lab hours (486 hours) are modeled as 2-hour weekly check-in sessions to represent off-campus internship monitoring.
    \item \textbf{Dynamic Gap Tolerance:} The optimality gap tolerance is now a user-configurable parameter (0.1\% - 20\%), allowing administrators to trade off between solution optimality and computation speed.
\end{itemize}

\subsection{Solution Strategies}

To address the NP-hard nature of the problem, two solution strategies were implemented. Both strategies support all three solvers (CBC, HiGHS, OR-Tools CP-SAT), creating a full 2$\times$3 matrix of six solver-strategy combinations:

\textbf{Program-by-Program Sequential Strategy:} Instead of solving the entire university schedule as a single monolithic problem, the system solves the schedule program-by-program (e.g., IT $\rightarrow$ IS), processing larger programs first. This approach includes a fairness mechanism that reserves a portion of prime time slots (9AM-3PM) for later programs, ensuring equitable distribution of desirable scheduling times. Occupied room-time slots from earlier programs are passed as blocked constraints to subsequent programs. This strategy can use any of the three solvers: CBC (MILP), HiGHS (MILP), or OR-Tools CP-SAT (Constraint Programming).

\textbf{Global Strategy:} The entire schedule is solved as a single optimization problem. While computationally more demanding, this approach can find globally optimal solutions. The Global strategy also supports all three solvers, though OR-Tools CP-SAT demonstrates significantly better performance on large instances compared to the MILP solvers.

\begin{algorithm}[H]
\caption{Program-by-Program Sequential Strategy}
\begin{algorithmic}[1]
\STATE \textbf{Initialize:} $OccupiedSlots \gets \emptyset$
\STATE $Programs \gets$ SortBySize(AllPrograms) \COMMENT{Largest first}
\FOR{each $program$ in $Programs$}
    \STATE $ReservedSlots \gets$ CalculateReservation(RemainingPrograms)
    \STATE $Model \gets$ BuildModel($program$)
    \STATE AddBlockedSlots($Model$, $OccupiedSlots \cup ReservedSlots$)
    \STATE $Solution \gets$ Solve($Model$, time=600s)
    \IF{$Solution$ is Feasible}
        \STATE $NewSlots \gets$ ExtractSlots($Solution$)
        \STATE $OccupiedSlots \gets OccupiedSlots \cup NewSlots$
        \STATE SaveSchedule($program$, $Solution$)
    \ELSE
        \RETURN Infeasible
    \ENDIF
\ENDFOR
\RETURN Success
\end{algorithmic}
\end{algorithm}

\section{Results and Discussion}

\subsection{Benchmarks and Guidelines}

Educational facility planning guidelines differ from one institution to another and depend on context. Cornell University's classroom space guidelines [7] state that large lecture halls with 151 or more seats should be used 43 percent of the time, while smaller classrooms with 1 - 100 seats should be used 59 percent of the time. Those rates allow for scheduling constraints plus flexibility. Strategic timetabling research [9] and systematic reviews [13] show that utilization rates of 75 - 85 percent are possible when institutions balance scheduling flexibility with limited resources. Our model uses the 75 - 85 percent range but also accepts input through CSV files so that it can be adapted to the resource limited conditions common in Philippine schools.

\subsection{Feasibility Analysis with Different Room Configurations}

The system includes a built-in feasibility checker that analyzes demand versus supply before attempting to solve. This allows administrators to experiment with different room configurations to find the optimal balance. Table \ref{tab:feasibility_configs} shows the feasibility analysis results for various lab-to-lecture room ratios.

\begin{table}[htbp]
\caption{Feasibility Analysis: Room Configuration Scenarios}
\label{tab:feasibility_configs}
\centering
\scriptsize
\begin{tabular}{lccccc}
\toprule
\textbf{Config} & \textbf{Labs} & \textbf{Lec} & \textbf{Lec Util.} & \textbf{Lab Util.} & \textbf{Status} \\
\midrule
Baseline & 6 & 11 & 64.1\% & 52.5\% & Feasible \\
More Labs & 8 & 9 & 78.3\% & 39.4\% & Feasible \\
More Lec & 4 & 13 & 54.2\% & 78.8\% & Feasible \\
Minimal & 4 & 8 & 88.1\% & 78.8\% & At Risk \\
\bottomrule
\end{tabular}
\end{table}

The baseline configuration (6 labs, 11 lecture rooms) provides comfortable buffer capacity with lecture utilization at 64.1\% and lab utilization at 52.5\%. The ``Minimal'' configuration demonstrates the threshold wherein the system approaches infeasibility, with utilization rates exceeding 75\% for both room types.

\subsection{Solver Performance Comparison}

In order to evaluate the system's performance, test cases were conducted using different solver configurations. The test dataset consists of 2 programs (IT and IS) with 18 total blocks: IT (3,3,3,2 blocks per year) and IS (2,2,2,1 blocks per year), covering 50 courses across both programs. The GUI allows users to select between solvers (CBC, HiGHS, OR-Tools CP-SAT) and strategies (Program-by-Program Sequential or Global).

\begin{table}[htbp]
\caption{Solver Performance Comparison (18 blocks, 50 courses, 17 rooms)}
\label{tab:solver_comparison}
\centering
\scriptsize
\begin{tabular}{lccccc}
\toprule
\textbf{Parameter} & \textbf{T1} & \textbf{T2} & \textbf{T3} & \textbf{T4} & \textbf{T5} \\
\midrule
Strategy & Seq & Seq & Seq & Global & Global \\
Solver & CP-SAT & HiGHS & CBC & CP-SAT & HiGHS \\
Time Limit & 120s/prog & 120s/prog & 120s/prog & 600s & 600s \\
\midrule
IT Solve Time & 26.9s & 45.2s & 52.1s & \multirow{2}{*}{37.0s} & \multirow{2}{*}{$>$600s} \\
IS Solve Time & 8.0s & 12.3s & 15.8s & & \\
\midrule
Total Time & 34.9s & 57.5s & 67.9s & 37.0s & Timeout \\
Conflicts & 0 & 0 & 0 & 0 & N/A \\
Status & Optimal & Optimal & Optimal & Optimal & Suboptimal \\
\bottomrule
\end{tabular}
\end{table}

\noindent\textit{Note: Seq = Program-by-Program Sequential Strategy. T1-T3 demonstrate the sequential strategy with different solvers, solving IT first (larger program), then IS. T4 uses Global strategy with CP-SAT. T5 shows that MILP solvers struggle with the global problem size. The 2$\times$3 matrix allows users to choose the best solver-strategy combination for their needs.}

The Program-by-Program Sequential strategy works well with every solver. OR-Tools CP-SAT (T1) reaches the best solution in less than 35 seconds. HiGHS (T2) and CBC (T3), both MILP solvers also complete the task when they follow the same step-by-step order - yet they need 57.5 s and 67.9 s. The decomposition therefore turns the task into something that ordinary MILP solvers can finish. The method first solves the larger program (IT, 11 blocks) then fits the smaller program (IS, 7 blocks) into the left over time slots - a fairness rule keeps 30\% of the prime interval 9 AM - 3 PM free for later programs.

OR-Tools CP-SAT with the Global strategy (T4) needs 37 seconds - modern constraint programming handles the whole problem in one step. HiGHS (T5), a traditional MILP solver, cannot reach the best solution within 600 seconds when it tackles the full model right away. The practical choices are therefore - use CP-SAT on the complete model or apply the sequential decomposition so that any solver can finish.

Figure~\ref{fig:solver_time} visualizes the solve time comparison across the different configurations tested in this study.

\begin{figure}[htbp]
\centering
\includegraphics[width=0.9\columnwidth]{fig_solver_time.png}
\caption{Solve time comparison for test dataset (50 courses, 18 blocks, 17 rooms). Program-Sequential and OR-Tools Global strategies achieve optimal solutions in under 40 seconds. Lower is better.}
\label{fig:solver_time}
\end{figure}

Figure~\ref{fig:utilization} compares the room utilization achieved by this study against benchmarks from related literature.

\begin{figure}[htbp]
\centering
\includegraphics[width=0.9\columnwidth]{fig_utilization.png}
\caption{Room utilization comparison. Cornell \cite{cornell2022classroom} recommends 43-59\% for large rooms. Austero et al. \cite{austero2022optimizing} achieved 75.8\% at Bicol University. Schininà \cite{schinina2024timetabling} achieved 72\% at University of Groningen. This study (Sequential Strategy) achieves a weighted average of 60.0\% utilization across all room types---intentionally below the 75-85\% target to maintain buffer capacity for projected enrollment growth. Higher utilization is not always better; excessive utilization ($>$85\%) reduces scheduling flexibility and leaves no room for unexpected changes.}
\label{fig:utilization}
\end{figure}

\subsection{Comparison with Manual Methods}

Qualitative comparison with traditional manual scheduling reveals several key advantages of the automated approach:

\begin{itemize}
    \item \textbf{Conflict-Free Guarantee:} The linear programming model enforces hard constraints that mathematically prevent double-booking, whereas manual methods often result in errors discovered only mid-semester.
    \item \textbf{Optimized Utilization:} The system targets 75-85\% room utilization, compared to typically lower rates achieved through manual ad-hoc assignment.
    \item \textbf{Rapid Re-scheduling:} When constraints change (e.g., room unavailability), the system can generate a new schedule in minutes rather than days.
    \item \textbf{Transparency:} The feasibility checker provides clear demand/supply metrics before solving, allowing informed decision-making.
\end{itemize}

\subsection{Simulation Configuration}

The GUI provides configurable simulation parameters for capacity planning and scenario analysis:

\begin{itemize}
    \item \textbf{Enrollment Override:} Administrators can specify the number of student blocks per year level for each program. For example, the test configuration used IT (3,3,3,2 blocks for Years 1-4) and IS (2,2,2,1 blocks), totaling 18 blocks with approximately 40 students per block.
    \item \textbf{Room Override:} The number of laboratory and lecture rooms can be adjusted independently. The baseline test used 6 lab rooms and 11 lecture rooms, providing 240 lab-hours and 440 lecture-hours of weekly capacity.
    \item \textbf{Semester Selection:} Toggle between Semester 1 and Semester 2 course offerings.
    \item \textbf{Solver and Strategy Selection:} Choose between CBC, HiGHS, or OR-Tools CP-SAT solvers, and between Program-by-Program Sequential or Global strategies.
\end{itemize}

Table~\ref{tab:test_config} summarizes the simulation configuration used for benchmarking.

\begin{table}[htbp]
\caption{Simulation Test Configuration}
\label{tab:test_config}
\centering
\scriptsize
\begin{tabular}{ll}
\toprule
\textbf{Parameter} & \textbf{Value} \\
\midrule
Programs & IT, IS \\
IT Blocks (Y1-Y4) & 3, 3, 3, 2 (11 total) \\
IS Blocks (Y1-Y4) & 2, 2, 2, 1 (7 total) \\
Total Blocks & 18 \\
Students per Block & 40 (avg) \\
Lab Rooms & 6 \\
Lecture Rooms & 11 \\
Total Courses & 50+ (Semester 1) \\
Time Window & 8:00 AM -- 5:00 PM \\
Lunch Break & 12:00 PM -- 1:00 PM \\
\bottomrule
\end{tabular}
\end{table}

\subsection{Dynamic Model Capabilities}

The CSV-based input design enables dynamic adaptation without code modifications:

\begin{itemize}
    \item \textbf{Enrollment Changes:} Modify \texttt{enrollment.csv} to add/remove student groups or adjust counts.
    \item \textbf{Course Updates:} Edit \texttt{courses.csv} to add new courses or change hour allocations.
    \item \textbf{Room Configuration:} Use the GUI's room override feature to test different lab/lecture ratios instantly.
\end{itemize}

This flexibility allows administrators to perform ``what-if'' analysis, such as projecting the impact of a 10\% enrollment increase or the temporary closure of a laboratory for maintenance.

\section{Conclusion and Future Work}

This study built and ran a binary integer linear program that assigns classrooms at CCMS. The program handles two problems - student numbers grow fast - yet the school owns few laboratory rooms. It works one program at a time starting with the largest plus sets aside time blocks for the smaller ones. The full data set needs about 300,000 yes-or-no variables. Python code calls the OR-Tools CP-SAT solver - it returns a perfect timetable in under 35 s and leaves no clashes.

The study shows how to tune standard optimization rules to fit Philippine higher education rules. One rule forces every laboratory class into a laboratory room - it may not use a lecture room. A second rule builds more than one valid meeting pattern for each course. Rather than lock the course into a single pattern, the solver picks the best from a short list, for example two meetings of 1.5 h or three meetings of 1 h. The objective is to place every class without overlap but also to use rooms sensibly.

Real data from the IT besides IS degrees of CCMS tested the model - it covered every year level, the 8:00 a.m. 5:00 p.m. window and the required lunch break. The outcome shows that free software can solve hard timetabling tasks even when money, rooms as well as time are scarce.

For future work the researchers plan to focus on four points. They will include faculty preferences and availability limits to raise satisfaction. They will extend the model so that it processes shared courses across different programs with less effort. They will build a web interface that non technical staff find easy to use. They will test machine learning methods that forecast the best parameter values from past data.

Philippine higher education institutions now serve more students each year - optimization models of this kind are vital for keeping education standards high. The study offers a scalable framework that other institutions may copy to get the most from limited resources and to keep classrooms and laboratories in full use.

\section*{Acknowledgment}

The authors thank the College of Computing and Multimedia Studies at Camarines Norte State College for providing access to enrollment and curriculum data.

\begin{thebibliography}{20}

\bibitem{alnaji2024optimizing}
L. Alnaji, S. M. Alsager, and O. Aymen, ``Optimizing faculty resource allocation in higher education: A mathematical model for strategic planning,'' \textit{International Journal of Advanced and Applied Sciences}, vol. 11, no. 9, pp. 88--99, 2024. [Online]. Available: \url{https://doi.org/10.21833/ijaas.2024.09.010}

\bibitem{gu2025}
X. Gu, M. Krish, S. Sohail, S. Thakur, F. Sabrina, and Z. Fan, ``From Integer Programming to Machine Learning: A Technical Review on Solving University Timetabling Problems,'' \textit{Computation}, vol. 13, no. 1, article 10, 2025. [Online]. Available: \url{https://doi.org/10.3390/computation13010010}

\bibitem{rabadia2025}
D. D. Rabadia, C. C. Calantoc, R. D. Torres, and M. L. Tan, ``University Course Timetabling: From Theory to Real-World Implementation---Extending Naderi's Framework with Real-World Constraints using Google OR-Tools (SCIP),'' Master of Data Analytics Group Project, University of Niagara Falls Canada, DAMO-610-3, Mar. 2025.

\bibitem{austero2022optimizing}
L. D. Austero, R. P. Medina, A. M. Sison, and J. B. Matias, ``Optimizing Faculty Workloads and Room Utilization using Heuristically Enhanced WOA,'' \textit{International Journal of Advanced Computer Science and Applications (IJACSA)}, vol. 13, no. 11, pp. 554--561, Nov. 2022. [Online]. Available: \url{https://thesai.org/Downloads/Volume13No11/Paper_64-Optimizing_Faculty_Workloads_and_Room_Utilization.pdf}

\bibitem{aygul2025}
Ö. Aygül, T. Hellgren, S. Azizi, and A. C. Trapp, ``A predict-and-prescribe framework for dynamic course scheduling toward strategic university scaling,'' \textit{Omega}, 2025, Article 103406. [Online]. Available: \url{https://doi.org/10.1016/j.omega.2025.103406}

\bibitem{bayudan2024expansions}
C. Bayudan-Dacuycuy, ``Expansions, quality, and affirmative action in public higher education institutions in the Philippines,'' Philippine Institute for Development Studies, Discussion Paper DPS 2024-01, 2024. [Online]. Available: \url{https://www.econstor.eu/bitstream/10419/311704/1/1917094426.pdf}

\bibitem{ceschia2023}
S. Ceschia, L. Di Gaspero, and A. Schaerf, ``Educational timetabling: Problems, benchmarks, and state-of-the-art results,'' \textit{European Journal of Operational Research}, vol. 308, no. 1, pp. 1--18, 2023. [Online]. Available: \url{https://doi.org/10.1016/j.ejor.2022.07.011}

\bibitem{ched2021stem}
Commission on Higher Education, ``Policies, Standards, and Guidelines for STEM Programs in Higher Education Institutions,'' CHED, Philippines, 2021. [Online]. Available: \url{https://ched.gov.ph/}

\bibitem{cornell2022classroom}
Cornell University, ``Classroom Space Guidelines,'' Division of Budget \& Planning, Feb. 2022. [Online]. Available: \url{https://dbp.cornell.edu/wp-content/uploads/2022/02/Classroom_Space_Guidelines_Feb_2022.pdf}

\bibitem{davison2024hybrid}
M. Davison, A. Kheiri, and K. G. Zografos, ``Modelling and solving the university course timetabling problem with hybrid teaching considerations,'' \textit{Journal of Scheduling}, vol. 28, no. 2, pp. 195--215, 2024. [Online]. Available: \url{https://doi.org/10.1007/s10951-024-00817-w}


\bibitem{lindahl2018strategic}
M. Lindahl, A. J. Mason, T. Stidsen, and M. Sørensen, ``A strategic view of University timetabling,'' \textit{European Journal of Operational Research}, vol. 266, no. 1, pp. 35--45, 2018. [Online]. Available: \url{https://doi.org/10.1016/j.ejor.2017.09.022}

\bibitem{mtonga2021}
K. Mtonga, E. Twahirwa, S. Kumaran, and K. Jayavel, ``Modelling Classroom Space Allocation at University of Rwanda: A Linear Programming Approach,'' \textit{Applications and Applied Mathematics: An International Journal (AAM)}, vol. 16, no. 1, article 40, 2021. [Online]. Available: \url{https://digitalcommons.pvamu.edu/aam/vol16/iss1/40}

\bibitem{navarro2022school}
A. M. Navarro, ``School Infrastructure in the Philippines: Where Are We Now and Where Should We Be Heading?,'' Philippine Institute for Development Studies, Discussion Paper DPS 2022-10, 2022. [Online]. Available: \url{https://pidswebs.pids.gov.ph/CDN/PUBLICATIONS/pidsdps2210.pdf}

\bibitem{ofcourse2023}
ofCourse, ``University Course Scheduling Software Pricing,'' 2023. [Online]. Available: \url{https://ofcourse.com/pricing/}

\bibitem{oude2019practices}
R. A. Oude Vrielink, E. A. Jansen, E. W. Hans, and J. van Hillegersberg, ``Practices in timetabling in higher education institutions: a systematic review,'' \textit{Annals of Operations Research}, vol. 275, no. 1, pp. 145--160, 2019. [Online]. Available: \url{https://doi.org/10.1007/s10479-017-2688-8}

\bibitem{psa2024digital}
Philippine Statistics Authority, ``Digital Economy Contributes 8.5 Percent to the Philippine Economy in 2024,'' PSA Press Release, Apr. 2025. [Online]. Available: \url{https://psa.gov.ph/content/digital-economy-contributes-85-percent-philippine-economy-2024}

\bibitem{schinina2024timetabling}
R. Schininà, ``Timetabling and Room Assignment at the University of Groningen: An Integer Linear Programming Approach,'' Bachelor's Thesis, University of Groningen, 2024. [Online]. Available: \url{https://fse.studenttheses.ub.rug.nl/33259/1/bMATH2024SchininaR.pdf}

\bibitem{spms2025}
Enhancing Strategic Performance Management System in Higher Education Institutions in the Philippines Through Digitalized e-SPMS. In: Hong, J.C. (ed.) \textit{New Technology in Education and Training (AEIT 2025)}. Lecture Notes in Educational Technology. Springer, Singapore, 2025. [Online]. Available: \url{https://doi.org/10.1007/978-981-96-8931-6_38}

\bibitem{tan2021}
J. S. Tan, S. L. Goh, G. Kendall, and N. R. Sabar, ``A survey of the state-of-the-art of optimisation methodologies in school timetabling problems,'' \textit{Expert Systems with Applications}, vol. 165, 2021, Art. no. 113943. [Online]. Available: \url{https://doi.org/10.1016/j.eswa.2020.113943}

\bibitem{vermuyten2018integrated}
H. Vermuyten, J. N. Rosa, I. Marques, J. Beliën, and G. Barbosa-Póvoa, ``Integrated staff scheduling at a medical emergency service: An optimisation approach,'' \textit{Expert Systems with Applications}, vol. 112, pp. 62--76, 2018. [Online]. Available: \url{https://doi.org/10.1016/j.eswa.2018.06.017}

\end{thebibliography}




\end{document}