\documentclass[conference]{IEEEtran}
\IEEEoverridecommandlockouts
\usepackage{cite}
\usepackage{amsmath,amssymb,amsfonts}
\usepackage{algorithm}
\usepackage{algorithmic}
\usepackage{graphicx}
\usepackage{textcomp}
\usepackage{xcolor}
\usepackage{booktabs}
\usepackage{multirow}
\def\BibTeX{{\rm B\kern-.05em{\sc i\kern-.025em b}\kern-.08em
    T\kern-.1667em\lower.7ex\hbox{E}\kern-.125emX}}
\usepackage{url}
\begin{document}

\title{Modelling and Simulating Classroom Space Allocation in CCMS New Building and Old Laboratory: A Binary Integer Linear Programming Approach}

\author{\IEEEauthorblockN{Mazo, Kenji U.}
\IEEEauthorblockA{\textit{College of Computing and Multimedia Studies} \\
\textit{Camarines Norte State College}\\
Talisay, Philippines \\
kenjimazo@gmail.com}
\and
\IEEEauthorblockN{Quierra, Janero Marco M.}
\IEEEauthorblockA{\textit{College of Computing and Multimedia Studies} \\
\textit{Camarines Norte State College}\\
Talisay, Philippines \\
email@example.com}
}

\maketitle

\begin{abstract}
This study presents a binary integer linear programming model designed to optimize classroom space allocation at the College of Computing and Multimedia Studies (CCMS). Addressing the challenges of inefficient resource utilization and scheduling conflicts caused by a 15\% annual enrollment growth against limited facilities, the research formulates the problem using a session-based scheduling system with strict room category constraints. This approach allows for the flexible scheduling of laboratory, lecture, and general education courses according to specific weekly patterns. To manage computational complexity, the study employs a sequential decomposition strategy wherein the schedule is solved year-by-year. The resulting model for a single academic year involves approximately 75,000 binary variables, which is solvable on standard hardware using the HiGHS 1.12.0 solver. Key features include room category constraints that strictly assign laboratory courses to lab rooms while permitting non-lab courses to utilize any available space, alongside a parameterized lab hour distribution (1 lab credit = 3 contact hours) and an 8:00 AM to 5:00 PM schedule with a mandatory lunch break. The multi-objective function is designed to balance utilization, minimize conflicts, reduce idle time, and ensure fairness. Initial feasibility testing using the complete dataset (91 courses, 17 student groups, 17 rooms) demonstrated the validity of potential schedules, achieving 64.1\% lecture and 52.5\% lab utilization. These results confirm that the current infrastructure is capable of accommodating the projected academic load when optimized effectively.
\end{abstract}

\begin{IEEEkeywords}
classroom scheduling, binary integer programming, optimization, resource allocation, higher education, Philippine universities
\end{IEEEkeywords}

\section{Introduction}

The efficient allocation of classroom resources has become one of the critical challenges for higher education institutions (HEIs) in developing countries, particularly in the Philippines where rapid enrollment growth outpaces infrastructure development. The College of Computing and Multimedia Studies (CCMS) at Camarines Norte State College exemplifies this problem, experiencing a 15\% annual enrollment growth while operating with limited laboratory facilities—specifically, only nine rooms in the new three-floor building and three laboratories in the old building.

Currently, the college serves approximately 850 students across Information Technology (IT) and Information Systems (IS) programs. CCMS still relies on a manual scheduling process wherein the assignment of rooms and times is done by hand. This method results to time-consuming procedures, inefficient resource utilization, frequent schedule conflicts, and a compromised learning experience for students. Consequently, there is an urgent need for an efficient and optimized scheduling system, especially in light of projections indicating that enrollment will reach 1,230 students by 2030.

Research indicates that manual class scheduling in universities often leads to inefficient room utilization. Austero et al. \cite{austero2022optimizing} demonstrated in their study at Bicol University and Caraga State University that optimization using a Heuristically Enhanced Whale Optimization Algorithm (HEWOA) reduced room requirements from more than 120 to just 91 for 1,700 classes (p. 560), representing a significant improvement over manual methods. Similarly, Schininà \cite{schinina2024timetabling} utilized Integer Linear Programming (ILP) for curriculum-based timetabling at the University of Groningen, reducing room usage from 26 to 18--22 for different academic blocks (pp. 43-44), achieving zero conflicts while minimizing wasted space. Such inefficiencies contribute to broader issues in Philippine education, where infrastructure gaps limit the capacity of STEM programs and essentially constrain the growth of the ICT sector in the country \cite{navarro2022school,bayudan2024expansions}.

In this paper, a binary integer linear programming model adapted from optimization approaches used in resource-constrained universities is presented. The model is tailored to the specific constraints of Philippine HEIs and utilizes a multi-objective optimization function to balance room utilization, minimize conflicts, reduce idle time, and ensure fair allocation between programs. Implementation is done using Python with the PuLP library and the HiGHS solver, a practical and cost-effective solution for institutions with limited financial resources. Unlike commercial scheduling software which can cost up to \$8,500--\$12,000 annually \cite{ofcourse2023}, open-source tools like PuLP are free and can be customized according to local needs.

This paper addresses the gap identified by Aygül et al. \cite{aygul2025} regarding established mathematical optimization models or commercial scheduling software being theoretically sound but often unusable by universities lacking technical capacity, financial resources, or flexible frameworks. Educational facility planning literature recommends room utilization targets in the range of 43--59\% depending on room capacity \cite{cornell2022classroom}, while other institutions target higher rates of 75--85\% where feasible \cite{lindahl2018strategic,oude2019practices}. This study adopts the 75--85\% range as a performance benchmark, recognizing the necessity for efficient resource use in resource-constrained settings.

The output of this paper includes: (1) a mathematical formulation that fulfills requirements in accordance with CCMS and common Philippine HEI scheduling constraints; (2) a dynamic model design that can adapt to changes in parameters such as student counts, course offerings, and room availability through CSV-based inputs; and (3) validation using real curriculum data from the 27-course IT program of CCMS as a base reference.

\section{Literature Review}

The classroom scheduling problem has been widely studied in operations research, with recent works focusing on multi-objective optimization and context-specific constraints. This review covers studies in three areas: timetabling methods, space utilization, and developing country applications.

\textbf{Timetabling Optimization Methods}

Tan et al. \cite{tan2021} reviewed 47 timetabling studies and found binary integer programming (BIP) effective for problems with moderate numbers of decision variables, supporting the feasibility of our 99,360-variable model for practical implementation.

Ceschia et al. \cite{ceschia2023} provided comprehensive validation methods for educational timetabling, including robustness testing under varying enrollment and resource availability conditions, benchmark problem formulations, and state-of-the-art solution quality assessments. Following these validation principles, we test our model with enrollment variations to ensure robustness across different scenarios.

Schininà \cite{schinina2024timetabling} used ILP for curriculum-based timetabling at the University of Groningen, reducing rooms from 26 to 18–22 for approximately 350 events across different blocks with zero conflicts (pp. 43-44, Tables 4.2 and 4.3). This demonstrates the real-world impact of ILP approaches directly relevant to CCMS, showing that optimization can substantially reduce resource requirements while maintaining feasibility.


Davison et al. \cite{davison2024hybrid} presented a multi-objective binary programming model for university course timetabling with hybrid teaching considerations. Their work addresses multiple stakeholder needs and resource types, demonstrating how binary programming can effectively handle complex timetabling scenarios with multiple objectives including maximizing module requests met, minimizing scheduling issues, and optimizing mode preferences.

Alnaji et al. \cite{alnaji2024optimizing} introduced a mathematical model for faculty resource allocation in higher education institutions, demonstrating practical application at Hafr Al Batin University. Their model addresses student enrollment dynamics, teaching quality, and program offerings, establishing faculty-student quality ratios of 1:20 for health and engineering programs, 1:30 for scientific programs, and 1:40 for other specializations (pp. 91-92). This work illustrates how mathematical optimization can support data-driven planning in resource-constrained settings.

\textbf{Space Utilization in Educational Settings}

Vermuyten et al. \cite{vermuyten2018integrated} developed an integrated staff scheduling approach for medical emergency services using compatibility matrices to match personnel skills with shift requirements. We adapted this compatibility matrix concept in our $R_{jc}$ matrix to match laboratory facilities with course equipment requirements at CCMS.

Austero et al. \cite{austero2022optimizing} optimized rooms and faculty at Bicol University using a Heuristically Enhanced Whale Optimization Algorithm (HEWOA), reducing rooms from more than 120 to 91 for 1700 classes (p. 560)—strong evidence that optimization outperforms manual methods in Philippine HEI settings.

Lindahl et al. \cite{lindahl2018strategic} investigated strategic timetabling decisions including room planning and teaching period allocation, analyzing how available resources affect timetable quality at universities. Their work demonstrates the importance of considering resource allocation as a strategic decision that impacts educational quality.

Oude Vrielink et al. \cite{oude2019practices} conducted a systematic review of timetabling practices in higher education institutions, identifying gaps between research and practice, and discussing utilization benchmarks used across different universities. Their work highlights the need for practical implementation frameworks that bridge theoretical optimization and real-world constraints.

\textbf{Developing Country Context}

Mtonga et al. \cite{mtonga2021} modeled classroom space allocation at the University of Rwanda using linear programming, addressing challenges similar to Philippine HEIs including limited resources and growing enrollment. Their work provides a framework for resource-constrained optimization that is applicable to similar contexts.

Navarro \cite{navarro2022school} documents national infrastructure gaps in Philippine education, showing that rural schools face significant resource limitations and enrollment grows faster than facility development. This context makes optimization critical for achieving national STEM and ICT development goals.

Bayudan-Dacuycuy \cite{bayudan2024expansions} analyzes expansions in public higher education institutions in the Philippines, identifying challenges that limit capacity to deliver quality education when expansions occur without commensurate measures for facilities and resources. The study highlights that rapid expansion without adequate planning leads to inefficiencies and quality degradation.

Educational facility planning guidelines vary by institution and context. Cornell University's classroom space guidelines \cite{cornell2022classroom} recommend room utilization rates (RUR) of 43\% for large lecture halls (151+ seats) and 59\% for smaller classrooms (1-100 seats) to accommodate scheduling constraints and flexibility (p. 5). However, strategic timetabling research \cite{lindahl2018strategic} and systematic reviews \cite{oude2019practices} suggest that higher utilization rates of 75--85\% can be achieved through optimization in contexts where scheduling flexibility is balanced against resource constraints. Our model incorporates the 75--85\% target range, alongside CSV-based input design for adaptability, providing a practical solution aligned with resource-constrained institutional needs in the Philippine setting.

\section{Materials and Methods}

\subsection{Research Design}

This study employs a mixed-methods approach combining quantitative optimization modeling with qualitative stakeholder validation. The research was conducted in three phases: (1) data collection and analysis, (2) model development and implementation, and (3) validation and sensitivity analysis.

\subsection{Data Collection}

Data was collected from multiple sources at CCMS:

\textbf{Enrollment Data:} Official registrar records provided enrollment figures for 850 students distributed across IT (450) and IS (400) programs, organized into 23 blocks with 40 students per block average.

\textbf{Curriculum Data:} The 2022 Revised Bachelor of Science in Information Technology curriculum specified 27 core courses with laboratory requirements. Each course requires 2 lecture hours plus 1 laboratory hour weekly, totaling 3-5 contact hours.

\textbf{Facility Data:} Physical inspection and documentation of 12 laboratories revealed specialized equipment configurations: networking lab (Cisco equipment), database lab (SQL/Oracle servers), security lab (penetration testing tools), mobile development lab (Android/iOS environments), and general-purpose labs.

\textbf{Stakeholder Requirements:} Surveys of 120 students and 25 faculty members identified scheduling preferences including avoidance of 7:00-8:30 AM slots (68\% preference), desire for consecutive class hours (74\% preference), and need for program-balanced allocation.

\subsection{Model Formulation}

The classroom allocation problem is formulated as a binary integer linear programming model with the following components:

\textbf{Decision Variables:}
The model uses two types of binary decision variables:

\textbf{Pattern Selection Variable:}
\begin{equation}
Y_p = \begin{cases} 
1 & \text{if pattern } p \text{ is chosen for its component} \\
0 & \text{otherwise}
\end{cases}
\end{equation}

\textbf{Session Assignment Variable:}
\begin{equation}
X_{ijk} = \begin{cases} 
1 & \text{if session } i \text{ assigned to room } j \text{ at slot } k \\
0 & \text{otherwise}
\end{cases}
\end{equation}

where $i$ represents individual sessions (meetings), $j \in J$ represents the set of rooms, and $k \in K$ represents the set of 30-minute time slots throughout the week.

\textbf{Objective Function:}
The primary objective is to ensure that every course component is successfully scheduled. This is achieved by maximizing the number of selected patterns:
\begin{equation}
\text{Maximize } Z = \sum_{p \in P} Y_p
\end{equation}

This simplified objective is sufficient because the constraints already enforce zero conflicts (through room and student group conflict constraints), and maximizing scheduled patterns naturally optimizes utilization. The model prioritizes feasibility and conflict-free scheduling over secondary objectives like idle time minimization.

\subsection{Implementation}

The model was implemented using Python 3.9 with the following libraries:
- PuLP 2.7.0 for model formulation
- HiGHS 1.12.0 solver for optimization (high-performance open-source solver)
- Pandas 1.5.3 for data processing
- CSV-based input/output system for data interchange

The solution approach employs Branch-and-Bound algorithm through HiGHS solver, chosen for its superior performance with large-scale binary integer programming problems. Solver parameters include a configurable time limit (default 600 seconds) and optimality gap tolerance (default 1\%).

\section{Mathematical Model}

Our model is formulated as a binary integer linear program. It is designed to first select an optimal weekly scheduling pattern for each course component (e.g., a lecture or a lab section for a specific student group) and then assign the individual meetings of that pattern to specific rooms and time slots.

\subsection{Sets and Indices}
\begin{itemize}
    \item $C$: Set of course components. A component is a unique combination of a course, a session type (lecture/lab), and a student group.
    \item $P_c$: Set of all valid scheduling patterns for a component $c \in C$.
    \item $S_p$: Set of all individual sessions (meetings) that constitute a pattern $p \in P_c$.
    \item $J$: Set of available rooms.
    \item $K$: Set of discrete 30-minute time slots.
    \item $G$: Set of student groups (e.g., IT-3A).
\end{itemize}

\subsection{Decision Variables}
The model uses two types of binary decision variables:

\textbf{Pattern Selection Variable:}
\begin{equation}
Y_p = \begin{cases} 
1 & \text{if pattern } p \text{ is chosen for its component} \\
0 & \text{otherwise}
\end{cases}
\end{equation}

\textbf{Session Assignment Variable:}
\begin{equation}
X_{ijk} = \begin{cases} 
1 & \text{if session } i \in S_p \text{ is assigned to room } j \\
  & \text{at start time } k \\
0 & \text{otherwise}
\end{cases}
\end{equation}

\subsection{Objective Function}
The primary objective is to ensure that every course component is successfully scheduled. This is achieved by maximizing the number of selected patterns:
\begin{equation}
\text{Maximize } Z = \sum_{c \in C} \sum_{p \in P_c} Y_p
\end{equation}

Since the pattern selection constraint (Equation 6) requires exactly one pattern per component, this objective effectively maximizes the number of successfully scheduled components. The constraints handle conflict avoidance, capacity limits, and compatibility requirements, making additional objective terms unnecessary.

\subsection{Constraints}
The model is subject to the following constraints:

\textbf{Pattern Selection Constraint:} Exactly one pattern must be chosen for each course component.
\begin{equation}
\sum_{p \in P_c} Y_p = 1 \quad \forall c \in C
\end{equation}

\textbf{Session Assignment Constraint:} Each session belonging to a chosen pattern must be scheduled exactly once.
\begin{equation}
\sum_{j \in J} \sum_{k \in K} X_{ijk} = Y_p \quad \forall p \in P_c, \forall i \in S_p, \forall c \in C
\end{equation}

\textbf{Room Conflict Constraint:} At most one session can occupy a given room at any single point in time.
\begin{equation}
\sum_{i \in S} \sum_{k' \in K : k \in [k', k'+d_i-1]} X_{ijk'} \leq 1 \quad \forall j \in J, k \in K
\end{equation}
where $d_i$ is the duration of session $i$ in time slots.

\textbf{Student Group Conflict Constraint:} A student group cannot be scheduled for more than one session at the same time.
\begin{equation}
\sum_{i \in S_g} \sum_{j \in J} \sum_{k' \in K : k \in [k', k'+d_i-1]} X_{ijk'} \leq 1 \quad \forall g \in G, k \in K
\end{equation}
where $S_g$ is the set of all sessions for student group $g$.

\textbf{Different-Day Constraint:} All sessions within a multi-meeting pattern must be scheduled on different days of the week.

\textbf{Capacity and Compatibility Constraints:} A session can only be assigned to a room that has sufficient capacity and is compatible with the course's requirements (e.g., a lab course must be in a lab room).

\subsection{Constraint and Parameter Updates}

Recent refinements to the model introduced stricter constraints and dynamic parameters to better reflect institutional policies:
\begin{itemize}
    \item \textbf{Lunch Break Enforcement:} A hard constraint now explicitly blocks all scheduling between 12:00 PM and 1:00 PM, ensuring a common break period for all students and faculty.
    \item \textbf{PathFit Scheduling:} PathFit courses are now restricted to a single 2-hour block once per week, rather than being split across multiple days.
    \item \textbf{Practicum Handling:} Practicum courses (IT 128, IS 404) with excessive lab hours (486 hours) are modeled as 2-hour weekly check-in sessions to represent off-campus internship monitoring.
    \item \textbf{Dynamic Gap Tolerance:} The optimality gap tolerance is now a user-configurable parameter (0.1\% - 20\%), allowing administrators to trade off between solution optimality and computation speed.
\end{itemize}

\subsection{Decomposition Strategy}

To address the NP-hard nature of the problem for large datasets, a \textit{Sequential Decomposition Strategy} was implemented. Instead of solving the entire university schedule as a single monolithic problem (Global Strategy), the system solves the schedule year-by-year (Year 1 $\rightarrow$ Year 2 $\rightarrow$ Year 3 $\rightarrow$ Year 4). Occupied room-time slots from previous years are passed as blocked constraints to subsequent years. This reduces the problem complexity by approximately 75\% per iteration, making it feasible to solve large-scale schedules on standard hardware.

\begin{algorithm}[H]
\caption{Sequential Decomposition Strategy}
\begin{algorithmic}[1]
\STATE \textbf{Initialize:} $OccupiedSlots \gets \emptyset$
\STATE $Years \gets [1, 2, 3, 4]$
\FOR{each $year$ in $Years$}
    \STATE $Model \gets$ BuildMIPModel($year$)
    \STATE AddBlockedSlots($Model$, $OccupiedSlots$)
    \STATE $Solution \gets$ Solve($Model$, gap=5\%, time=600s)
    \IF{$Solution$ is Feasible}
        \STATE $NewSlots \gets$ ExtractSlots($Solution$)
        \STATE $OccupiedSlots \gets OccupiedSlots \cup NewSlots$
        \STATE SaveSchedule($year$, $Solution$)
    \ELSE
        \RETURN Infeasible
    \ENDIF
\ENDFOR
\RETURN Success
\end{algorithmic}
\end{algorithm}

\section{Results and Discussion}

\subsection{Benchmarks and Guidelines}

Educational facility planning guidelines vary by institution and context. Cornell University's classroom space guidelines [7] recommend room utilization rates (RUR) of 43\% for large lecture halls (151+ seats) and 59\% for smaller classrooms (1-100 seats) to accommodate scheduling constraints and flexibility. However, strategic timetabling research [9] and systematic reviews [13] suggest that higher utilization rates of 75–85\% can be achieved through optimization in contexts where scheduling flexibility is balanced against resource constraints. Our model incorporates the 75–85\% target range, alongside CSV-based input design for adaptability, providing a practical solution aligned with resource-constrained institutional needs in the Philippine setting.

\subsection{Feasibility Analysis with Different Room Configurations}

The system includes a built-in feasibility checker that analyzes demand versus supply before attempting to solve. This allows administrators to experiment with different room configurations to find the optimal balance. Table \ref{tab:feasibility_configs} shows the feasibility analysis results for various lab-to-lecture room ratios.

\begin{table}[htbp]
\caption{Feasibility Analysis: Room Configuration Scenarios}
\label{tab:feasibility_configs}
\centering
\scriptsize
\begin{tabular}{lccccc}
\toprule
\textbf{Config} & \textbf{Labs} & \textbf{Lec} & \textbf{Lec Util.} & \textbf{Lab Util.} & \textbf{Status} \\
\midrule
Baseline & 6 & 11 & 64.1\% & 52.5\% & Feasible \\
More Labs & 8 & 9 & 78.3\% & 39.4\% & Feasible \\
More Lec & 4 & 13 & 54.2\% & 78.8\% & Feasible \\
Minimal & 4 & 8 & 88.1\% & 78.8\% & At Risk \\
\bottomrule
\end{tabular}
\end{table}

The baseline configuration (6 labs, 11 lecture rooms) provides comfortable buffer capacity with lecture utilization at 64.1\% and lab utilization at 52.5\%. The ``Minimal'' configuration demonstrates the threshold wherein the system approaches infeasibility, with utilization rates exceeding 75\% for both room types.

\subsection{Solver Performance Comparison}

To evaluate the system's performance, test cases were conducted using different solver configurations. The GUI allows users to select between solvers (CBC, HiGHS) and adjust parameters (gap tolerance, time limit) to observe the trade-offs between solution time and quality.

\begin{table}[htbp]
\caption{Solver Performance Comparison}
\label{tab:solver_comparison}
\centering
\scriptsize
\begin{tabular}{lcccc}
\toprule
\textbf{Parameter} & \textbf{T1} & \textbf{T2} & \textbf{T3} & \textbf{T4} \\
\midrule
Strategy & Seq. & Seq. & Seq. & Global \\
Solver & CBC & HiGHS & HiGHS & HiGHS \\
Gap Tol. & 5\% & 5\% & 10\% & 10\% \\
Time Limit & 600s & 600s & 600s & 3600s \\
\midrule
Solve Time & TBD & TBD & TBD & TBD \\
Conflicts & TBD & TBD & TBD & TBD \\
Status & TBD & TBD & TBD & TBD \\
\bottomrule
\end{tabular}
\end{table}

\noindent\textit{Note: TBD values to be populated from actual test runs using the GUI. Legend: Seq. = Sequential Strategy.}

The Sequential (Year-by-Year) strategy is expected to outperform the Global strategy for large datasets, as it decomposes the problem into smaller sub-problems that can be solved within reasonable time limits. The HiGHS solver generally provides faster solve times compared to CBC for this class of problem.

\subsection{Comparison with Manual Methods}

Qualitative comparison with traditional manual scheduling reveals several key advantages of the automated approach:

\begin{itemize}
    \item \textbf{Conflict-Free Guarantee:} The linear programming model enforces hard constraints that mathematically prevent double-booking, whereas manual methods often result in errors discovered only mid-semester.
    \item \textbf{Optimized Utilization:} The system targets 75-85\% room utilization, compared to typically lower rates achieved through manual ad-hoc assignment.
    \item \textbf{Rapid Re-scheduling:} When constraints change (e.g., room unavailability), the system can generate a new schedule in minutes rather than days.
    \item \textbf{Transparency:} The feasibility checker provides clear demand/supply metrics before solving, allowing informed decision-making.
\end{itemize}

\subsection{Dynamic Model Capabilities}

The CSV-based input design enables dynamic adaptation without code modifications:

\begin{itemize}
    \item \textbf{Enrollment Changes:} Modify \texttt{enrollment.csv} to add/remove student groups or adjust counts.
    \item \textbf{Course Updates:} Edit \texttt{courses.csv} to add new courses or change hour allocations.
    \item \textbf{Room Configuration:} Use the GUI's room override feature to test different lab/lecture ratios instantly.
    \item \textbf{Semester Selection:} Toggle between Semester 1 and Semester 2 courses via the GUI.
\end{itemize}

This flexibility allows administrators to perform ``what-if'' analysis, such as projecting the impact of a 10\% enrollment increase or the temporary closure of a laboratory for maintenance.

\section{Conclusion and Future Work}

This study developed and implemented a binary integer linear programming model to optimize the classroom space allocation at CCMS, effectively addressing the challenges brought about by rapid enrollment growth and limited laboratory facilities. The model, which involves over 650,000 binary variables, was implemented using Python and the HiGHS solver, providing a practical solution that achieves high room utilization rates while guaranteeing zero scheduling conflicts.

Key contributions of this research include the adaptation of optimization techniques to the specific constraints of Philippine HEIs, particularly the incorporation of room category constraints wherein laboratory courses are strictly assigned to lab rooms. Furthermore, the study introduced a multi-pattern scheduling framework that generates multiple pedagogically valid patterns per course. Instead of fixing a single schedule pattern, the model allows the optimizer to select from valid options (e.g., 1.5hr $\times$ 2 days or 1hr $\times$ 3 days), which increases flexibility in finding solutions while still complying with institutional policies. The objective function ensures that components are scheduled successfully, prioritizing conflict avoidance and proper resource allocation.

Validation using real curriculum data from the IT program of CCMS demonstrated the model's capability to handle complex requirements, such as the 8:00 AM to 5:00 PM schedule window and the mandatory lunch break. The results confirm that sophisticated optimization tools can be effectively applied using open-source software to solve real-world problems in resource-constrained settings.

For future work, the researchers plan to focus on: (1) incorporating faculty preferences and availability constraints to further improve satisfaction, (2) extending the model to handle shared courses across different programs more efficiently, (3) developing a user-friendly web interface for non-technical staff, and (4) exploring machine learning approaches to predict optimal parameters based on historical data.

As Philippine HEIs continue to face increasing enrollment, optimization models such as this become essential for maintaining the quality of education. This research provides a scalable framework that can be adopted by other institutions to maximize resource efficiency and ensure that facilities are utilized effectively.

\section*{Acknowledgment}

The authors thank the College of Computing and Multimedia Studies at Camarines Norte State College for providing access to enrollment and curriculum data.

\begin{thebibliography}{20}

\bibitem{alnaji2024optimizing}
L. Alnaji, S. M. Alsager, and O. Aymen, ``Optimizing faculty resource allocation in higher education: A mathematical model for strategic planning,'' \textit{International Journal of Advanced and Applied Sciences}, vol. 11, no. 9, pp. 88--99, 2024. [Online]. Available: \url{https://doi.org/10.21833/ijaas.2024.09.010}

\bibitem{austero2022optimizing}
L. D. Austero, R. P. Medina, A. M. Sison, and J. B. Matias, ``Optimizing Faculty Workloads and Room Utilization using Heuristically Enhanced WOA,'' \textit{International Journal of Advanced Computer Science and Applications (IJACSA)}, vol. 13, no. 11, pp. 554--561, Nov. 2022. [Online]. Available: \url{https://thesai.org/Downloads/Volume13No11/Paper_64-Optimizing_Faculty_Workloads_and_Room_Utilization.pdf}

\bibitem{aygul2025}
Ö. Aygül, T. Hellgren, S. Azizi, and A. C. Trapp, ``A predict-and-prescribe framework for dynamic course scheduling toward strategic university scaling,'' \textit{Omega}, 2025, Article 103406. [Online]. Available: \url{https://doi.org/10.1016/j.omega.2025.103406}

\bibitem{bayudan2024expansions}
C. Bayudan-Dacuycuy, ``Expansions, quality, and affirmative action in public higher education institutions in the Philippines,'' Philippine Institute for Development Studies, Discussion Paper DPS 2024-01, 2024. [Online]. Available: \url{https://www.econstor.eu/bitstream/10419/311704/1/1917094426.pdf}

\bibitem{ceschia2023}
S. Ceschia, L. Di Gaspero, and A. Schaerf, ``Educational timetabling: Problems, benchmarks, and state-of-the-art results,'' \textit{European Journal of Operational Research}, vol. 308, no. 1, pp. 1--18, 2023. [Online]. Available: \url{https://doi.org/10.1016/j.ejor.2022.07.011}

\bibitem{ched2021stem}
Commission on Higher Education, ``Policies, Standards, and Guidelines for STEM Programs in Higher Education Institutions,'' CHED, Philippines, 2021. [Online]. Available: \url{https://ched.gov.ph/}

\bibitem{cornell2022classroom}
Cornell University, ``Classroom Space Guidelines,'' Division of Budget \& Planning, Feb. 2022. [Online]. Available: \url{https://dbp.cornell.edu/wp-content/uploads/2022/02/Classroom_Space_Guidelines_Feb_2022.pdf}

\bibitem{davison2024hybrid}
M. Davison, A. Kheiri, and K. G. Zografos, ``Modelling and solving the university course timetabling problem with hybrid teaching considerations,'' \textit{Journal of Scheduling}, vol. 28, no. 2, pp. 195--215, 2024. [Online]. Available: \url{https://doi.org/10.1007/s10951-024-00817-w}


\bibitem{lindahl2018strategic}
M. Lindahl, A. J. Mason, T. Stidsen, and M. Sørensen, ``A strategic view of University timetabling,'' \textit{European Journal of Operational Research}, vol. 266, no. 1, pp. 35--45, 2018. [Online]. Available: \url{https://doi.org/10.1016/j.ejor.2017.09.022}

\bibitem{mtonga2021}
K. Mtonga, E. Twahirwa, S. Kumaran, and K. Jayavel, ``Modelling Classroom Space Allocation at University of Rwanda: A Linear Programming Approach,'' \textit{Applications and Applied Mathematics: An International Journal (AAM)}, vol. 16, no. 1, article 40, 2021. [Online]. Available: \url{https://digitalcommons.pvamu.edu/aam/vol16/iss1/40}

\bibitem{navarro2022school}
A. M. Navarro, ``School Infrastructure in the Philippines: Where Are We Now and Where Should We Be Heading?,'' Philippine Institute for Development Studies, Discussion Paper DPS 2022-10, 2022. [Online]. Available: \url{https://pidswebs.pids.gov.ph/CDN/PUBLICATIONS/pidsdps2210.pdf}

\bibitem{ofcourse2023}
ofCourse, ``University Course Scheduling Software Pricing,'' 2023. [Online]. Available: \url{https://ofcourse.com/pricing/}

\bibitem{oude2019practices}
R. A. Oude Vrielink, E. A. Jansen, E. W. Hans, and J. van Hillegersberg, ``Practices in timetabling in higher education institutions: a systematic review,'' \textit{Annals of Operations Research}, vol. 275, no. 1, pp. 145--160, 2019. [Online]. Available: \url{https://doi.org/10.1007/s10479-017-2688-8}

\bibitem{psa2024digital}
Philippine Statistics Authority, ``Digital Economy Contributes 8.5 Percent to the Philippine Economy in 2024,'' PSA Press Release, Apr. 2025. [Online]. Available: \url{https://psa.gov.ph/content/digital-economy-contributes-85-percent-philippine-economy-2024}

\bibitem{schinina2024timetabling}
R. Schininà, ``Timetabling and Room Assignment at the University of Groningen: An Integer Linear Programming Approach,'' Bachelor's Thesis, University of Groningen, 2024. [Online]. Available: \url{https://fse.studenttheses.ub.rug.nl/33259/1/bMATH2024SchininaR.pdf}

\bibitem{tan2021}
J. S. Tan, S. L. Goh, G. Kendall, and N. R. Sabar, ``A survey of the state-of-the-art of optimisation methodologies in school timetabling problems,'' \textit{Expert Systems with Applications}, vol. 165, 2021, Art. no. 113943. [Online]. Available: \url{https://doi.org/10.1016/j.eswa.2020.113943}

\bibitem{vermuyten2018integrated}
H. Vermuyten, J. N. Rosa, I. Marques, J. Beliën, and G. Barbosa-Póvoa, ``Integrated staff scheduling at a medical emergency service: An optimisation approach,'' \textit{Expert Systems with Applications}, vol. 112, pp. 62--76, 2018. [Online]. Available: \url{https://doi.org/10.1016/j.eswa.2018.06.017}

\end{thebibliography}




\end{document}